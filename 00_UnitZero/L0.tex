\documentclass[11pt]{article}

% This line looks for the preamble in the folder "next door"
% It goes UP one level (..) then DOWN into (common)
\input{CommonPreamble/CommonPreamble}

\chead{00 : Intro and Review }

\begin{document}

\section{Review of Differential Equations}
\subsection{Seperation of variables}
General First Order ODE:
\begin{align*}
\frac{dy}{dt}+py &= g
\end{align*}
If p and g are constants we can solve this by seperation of variables:

\begin{align*}
    \frac{dy}{dt} + py &= g \\
    \frac{dy}{dt} &= g - py \\
    \frac{1}{g - py} dy &= dt \\
    \intertext{Integrate both sides:}
    \int \frac{1}{g - py} dy &= \int 1 dt \\
    \intertext{\textcolor{blue}{Substitution: Let $u = g - py \implies du = -p dy$}
    \intertext{\textcolor{blue}{$\implies dy = -\frac{1}{p}du$}}}
    \int \frac{1}{u} \left( -\frac{1}{p} \right) du &= t + C \\
    -\frac{1}{p} \ln|u| &= t + C \\
    \ln|u| &= -p(t + C) \\
    \ln|u| &= -pt + C_{new} \quad \text{(where } C_{new} = -pC \text{)} \\
    |u| &= e^{-pt + C_{new}} \\
    u &= \pm e^{C_{new}} e^{-pt} \\
    \intertext{Let $C_1 = \pm e^{C_{new}}$. Substitute back $u = g - py$:}
    g - py &= C_1 e^{-pt} \\
    -py &= C_1 e^{-pt} - g \\
    y &= \frac{C_1 e^{-pt}}{-p} + \frac{-g}{-p} \\
    \boxed{y(t) = C_2 e^{-pt} + \frac{g}{p}} &\quad \text{where } C_2 = -\frac{C_1}{p}
\end{align*}

\subsubsection*{Behavior Analysis}
The solution consists of two distinct parts:
\begin{equation*}
    y(t) = \underbrace{C_2 e^{-pt}}_{\text{Transient Solution}} + \underbrace{\frac{g}{p}}_{\text{Steady State Solution}}
\end{equation*}
\begin{itemize}
    \item \textbf{Transient:} If $p > 0$, this term decays to zero as $t \to \infty$.
    \item \textbf{Steady State:} The value $y(t)$ approaches $\frac{g}{p}$ as time goes on.
    \item \textbf{Stability:} If $p < 0$, the exponential term grows to infinity (Unstable).
\end{itemize}

\subsubsection*{Solving for the Initial Condition}
Given an initial value $y(0) = y_0$, we can solve for $C_2$:
\begin{align*}
    y(0) &= C_2 e^{-p(0)} + \frac{g}{p} \\
    y_0 &= C_2 (1) + \frac{g}{p} \\
    C_2 &= y_0 - \frac{g}{p}
\end{align*}
Substituting this back into the general solution gives the specific solution:
\begin{equation*}
    \boxed{y(t) = \left( y_0 - \frac{g}{p} \right) e^{-pt} + \frac{g}{p}}
\end{equation*}
Remember, this is for a LINEAR 1ST ORDER ODE WITH CONSTANT COEFFICIENTS.

\clearpage
\subsection{Method of integrating Factors}

\begin{align*}
    \tikzmarknode{eq}{\frac{dy}{dt}} + p(t)y &= \tikzmarknode{qt}{q(t)} 
    \end{align*}

Rewrite to match the form of :
\begin{align*}
    \tikzmarknode{matchE}{\frac{d}{dt}}\left(\mu(t)y\right)&=\mu(t)q(t)\\
    \int\frac{d}{dt}\left(\mu(t)y\right)dt'&=\int_{0}^{T}\mu(t)q(t)dt'\\
    \mu(t)y(t)-\mu(0)y(0)&=\int_{0}^{T}\mu(t')q(t')dt'\\
    \mu(t)y(t)&=\int_{0}^{T}\mu(t')q(t')dt'+\mu(0)y(0)\\
    y(t)&=\frac{1}{\mu(t)}\Bigg[\int_{0}^{T}\mu(t')q(t')dt'+\mu(0)y(0)\Bigg]\\
\end{align*}
To find $\mu$, we match both sides of the equation.
\begin{align*}
    \tikzmarknode{matchS}{\frac{d}{dt}}\Bigg(\mu(t)y\Bigg)&=\frac{d\mu}{dt}y+\mu \frac{dy}{dt} \tag{$\bigstar$}
\end{align*}
Multiply the original equation by $\mu$:
\begin{align*}
    \tikzmarknode{originals}{\mu}\Bigg(\frac{dy}{dt}+p(t)y&=q(t) \Bigg) \\
    \mu\frac{dy}{dt}+\mu p(t)y&=\mu q(t) \tag{$\Delta$}
\end{align*}
Now we can compare $\bigstar$ and $\Delta$

\begin{align*}
    \frac{d\mu}{dt}&=\mu p(t)\\
    \frac{1}{\mu}d\mu&=p(t)dt\\
    \mu&=\boxed{Ce^{\int p(t)dt}}
\end{align*}





\begin{tikzpicture}[overlay, remember picture]
    \draw[<-, >=Latex, red, thick] (qt.south) 
        |- ++(0.5, -0.8) % Moves down 0.8cm and right 0.5cm
        node[right, text=black, font=\small] {Non Homog due to $q(t)$};

         \draw[->, >=Latex, blue, thick, rounded corners] (matchS.west) 
        -- ++(-2.8, 0) coordinate(turnPoint)      % Move Left 3cm, save this spot as 'turnPoint'
        -- (turnPoint |- matchE.west)  % Go Up (keep x of turnPoint, get y of matchE)
        -- (matchE.west);  % Go Right to the target
        \draw[->, >=Latex, red, thick, rounded corners] (originals.west) 
        -- ++(-3.2, 0) coordinate(turnPoint2)      % Move Left 3cm, save this spot as 'turnPoint'
        -- (turnPoint2 |- eq.west)  
        -- (eq.west);  % Go Right to the target
\end{tikzpicture}

\subsubsection*{Note on the Constant $C$}
\begin{align*}
    &\text{We can choose } C=1 \text{ because any constant cancels out.} \\
    &\text{why:} \\
    &\text{Let } \mu = C e^{\int p(t)dt}. \text{ Substitute into general solution:} \\
    y(t) &= \frac{1}{C e^{\int p dt}} \Bigg[ \int C e^{\int p dt} q(t) dt + C_{int} \Bigg] \\
    &\text{Factor out } C \text{ from the integral:} \\
    y(t) &= \frac{1}{C e^{\int p dt}} \cdot C \Bigg[ \int e^{\int p dt} q(t) dt + \frac{C_{int}}{C} \Bigg] \\
    &\text{Cancel the } C \text{ terms:} \\
    y(t) &= \frac{1}{e^{\int p dt}} \Bigg[ \int e^{\int p dt} q(t) dt + C_{new} \Bigg] \\
   &\text{ The constant } C \text{ in } \mu \text{ is redundant.}
\end{align*}

\subsubsection*{Big Must: Standard Form}
\begin{align*}
    &\text{The derivation assumes the coefficient of } \frac{dy}{dt} \text{ is } \mathbf{1}. \\
    &\text{If given a general form:} \\
    &a(t)\frac{dy}{dt} + b(t)y = g(t) \\
    &\text{You \textbf{must} divide by } a(t) \text{ first:} \\
    &\frac{dy}{dt} + \frac{b(t)}{a(t)}y = \frac{g(t)}{a(t)} \\
    &\text{Identify your terms:} \\
    &p(t) = \frac{b(t)}{a(t)}, \quad q(t) = \frac{g(t)}{a(t)}
\end{align*}

\subsubsection*{Algorithm Summary}
\begin{align*}
    1. \quad &\textbf{Standardize:} \\
    &\text{Ensure } \frac{dy}{dt} \text{ has a coefficient of 1.} \\
    2. \quad &\textbf{Identify:} \\
    &\text{Find } p(t) \text{ and } q(t). \\
    3. \quad &\textbf{Calculate } \boldsymbol{\mu}: \\
    &\mu(t) = e^{\int p(t)dt} \quad (\text{ignore } +C). \\
    4. \quad &\textbf{Solve:} \\
    &y(t) = \frac{1}{\mu(t)} \Bigg[ \int \mu(t)q(t) dt + C \Bigg].
\end{align*}

\clearpage
\subsection{2nd Order Linear Differential Equations}
A general form of a 2nd order ODE:
\begin{align*}
    \frac{d^2y}{dt^2}&= f\Bigg(t,y,\frac{dy}{dt}\Bigg)\\
    &\textbf{or}\\
    F\Bigg(t,y,\frac{dy}{dt},\frac{d^2y}{dt^2}\Bigg)&=0
\end{align*}

\subsection*{A quick note of Implicit, Explicit and Homogeneity}
When we say $F\Bigg(x,y,PB,Jelly,t,\frac{dy}{dt},\frac{d^2y}{dt^2}\Bigg)$, the \textbf{Container} F holds all combinations of said \\
$x,y,PB,Jelly,t,\frac{dy}{dt},\frac{d^2y}{dt^2}$ that add add to \textcolor{red}{0}.
Please note, this doesn't mean its \textcolor{red}{Homogeneous} in how its presented here. More on this in a little.\\
\\
For the other form:
\begin{align*}
    \frac{d^2y}{dt^2}&= f\Bigg(t,y,\frac{dy}{dt}\Bigg)
\end{align*}
Here we are saying, "if you give me current time, position and velocity, Ill be able to calculate acceleration"

Back to \textbf{Homgeneous or not}.\\
In the form $F\Bigg(t,y,\frac{dy}{dt},\frac{d^2y}{dt^2}\Bigg)=0$ we have only moved all things in the container to one side. Using my ridiculous example from before,\\
\\
Imagine an Implicit Function $F$ representing the ingredients of a lunch equation:
\begin{align*}
    F\Bigg(x, y, \text{PB}, \text{Jelly}, t, \frac{dy}{dt}, \frac{d^2y}{dt^2}\Bigg) &= 0
\end{align*}
This is \textbf{Implicit} because the ingredients are all mixed together in the bucket $F$.

To check if it is \textbf{Homogeneous}, we try to isolate the sandwich mechanics ($y, y', y''$) from the external ingredients:
\begin{align*}
    \underbrace{\frac{d^2y}{dt^2} + p(t)\frac{dy}{dt} + q(t)y}_{\text{The Bread}} &= \underbrace{\text{PB} + \text{Jelly}}_{\text{External Forcing Function } f(t)}
\end{align*}
\begin{itemize}
    \item If $\text{PB} = 0$ and $\text{Jelly} = 0$, the equation is \textbf{Homogeneous} (Just Bread).
    \item If $\text{PB} \neq 0$ or $\text{Jelly} \neq 0$, the equation is \textbf{Non-Homogeneous} (A Sandwich exists).
\end{itemize}


\end{document}

