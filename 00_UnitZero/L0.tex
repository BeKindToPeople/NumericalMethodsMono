\documentclass[11pt]{article}

% This line looks for the preamble in the folder "next door"
% It goes UP one level (..) then DOWN into (common)
\input{CommonPreamble/CommonPreamble}

\chead{00 : Intro and Review }

\begin{document}

\section{Review of Differential Equations}
\subsection{Seperation of variables}
General First Order ODE:
\begin{align*}
\frac{dy}{dt}+py &= g
\end{align*}
If p and g are constants we can solve this by seperation of variables:

\begin{align*}
    \frac{dy}{dt}+py&=g\\
    \frac{dy}{dt}&= g-py\\
    \frac{1}{g-py}dy&=1dt\\
    \int{\frac{1}{g-py}}dy&=\int1dt\\
    &\textcolor{blue}{u=g-py}\\
    &\textcolor{blue}{du=-pdy\rightarrow dy=-\frac{du}{p}}\\
    \int\frac{1}{u}\frac{-1}{p}du&=\int dt\\
    -\frac{1}{p}ln|u|&=t+C\\
    ln|u|&=(t+C)(-p)\\
    ln|u|&=-tp+C\\
    u&=e^{-tp+C}\\
    &=e^{-tp}e^{C}\\
    u&=e^{-tp}\textcolor{teal}{\mathbf{C_1}}\\
    g-py&=e^{-tp}C_1\\
    -py&=e^{-tp}C_1-g\\
    y&=-\frac{e^{-tp}C_1-g}{p}\\
    y&=-\frac{e^{-tp}C_1}{p}+\frac{g}{p}\\
    y&=e^{-tp}\textcolor{teal}{\mathbf{C_2}}+\frac{g}{p}
    \end{align*}
Here $C_1$ and $C_2$ are arbitrary constants. We will use those for Initial Value problems (IVP) in the future.

\clearpage
\subsection{Method of integrating Factors}

\begin{align*}
    \tikzmarknode{eq}{\frac{dy}{dt}} + p(t)y &= \tikzmarknode{qt}{q(t)} 
    \end{align*}

Rewrite to match the form of :
\begin{align*}
    \tikzmarknode{matchE}{\frac{d}{dt}}\left(\mu(t)y\right)&=\mu(t)q(t)\\
    \int\frac{d}{dt}\left(\mu(t)y\right)dt'&=\int_{0}^{T}\mu(t)q(t)dt'\\
    \mu(t)y(t)-\mu(0)y(0)&=\int_{0}^{T}\mu(t')q(t')dt'\\
    \mu(t)y(t)&=\int_{0}^{T}\mu(t')q(t')dt'+\mu(0)y(0)\\
    y(t)&=\frac{1}{\mu(t)}\Bigg[\int_{0}^{T}\mu(t')q(t')dt'+\mu(0)y(0)\Bigg]\\
\end{align*}
To find $\mu$, we match both sides of the equation.
\begin{align*}
    \tikzmarknode{matchS}{\frac{d}{dt}}\Bigg(\mu(t)y\Bigg)&=\frac{d\mu}{dt}y+\mu \frac{dy}{dt} \tag{$\bigstar$}
\end{align*}
Multiply the original equation by $\mu$:
\begin{align*}
    \tikzmarknode{originals}{\mu}\Bigg(\frac{dy}{dt}+p(t)y&=q(t) \Bigg) \\
    \mu\frac{dy}{dt}+\mu p(t)y&=\mu q(t) \tag{$\Delta$}
\end{align*}
Now we can compare $\bigstar$ and $\Delta$

\begin{align*}
    \frac{d\mu}{dt}&=\mu p(t)\\
    \frac{1}{\mu}d\mu&=p(t)dt\\
    \mu&=\boxed{Ce^{\int p(t)dt}}
\end{align*}




\begin{tikzpicture}[overlay, remember picture]
    \draw[<-, >=Latex, red, thick] (qt.south) 
        |- ++(0.5, -0.8) % Moves down 0.8cm and right 0.5cm
        node[right, text=black, font=\small] {Non Homog due to $q(t)$};

         \draw[->, >=Latex, blue, thick, rounded corners] (matchS.west) 
        -- ++(-2.8, 0) coordinate(turnPoint)      % Move Left 3cm, save this spot as 'turnPoint'
        -- (turnPoint |- matchE.west)  % Go Up (keep x of turnPoint, get y of matchE)
        -- (matchE.west);  % Go Right to the target
        \draw[->, >=Latex, red, thick, rounded corners] (originals.west) 
        -- ++(-3.2, 0) coordinate(turnPoint2)      % Move Left 3cm, save this spot as 'turnPoint'
        -- (turnPoint2 |- eq.west)  
        -- (eq.west);  % Go Right to the target
\end{tikzpicture}


\end{document}

